\addchap*{Abstract}
Recent years have seen a significant advancement in autonomous driving technology, which has great promise for improving both the efficiency and safety of transportation. However, maintaining autonomous vehicles' dependability and security is still a major problem. Autonomous driving vehicles distinguish the objects in the surroundings with the help of pre-trained neural network models and sensor data. Training a highly accurate deep neural model requires the model to see a huge amount of unbiased annotated data. In the context of lidar sensor data, it is very laborious and difficult to get annotated training data. Sota datasets like semantic kitti are publicly available to forward the development of \acrfull{had} system research in tasks like semantic segmentation, and object detection. However, because of the vast amount of interacting possibilities between the environment and objects, publicly available datasets like semantic kitti is not enough to test about the functionality of \acrshort{had} systems. Researchers and practitioners also decided on synthetic data from simulations. Because of the domain gaps between the simulations and real world, even though the performance of the models like object detection improved, these are still not adequate.

The thesis deals with the augmentation of lidar data. The approach followed involves the extraction of objects by using the 3d-geometric information of points from a point cloud source and enhancing another point cloud target by placing the object on the target. This thesis presents a simple yet effective approach to creating new scenario point cloud data. These methods could be used in testing with real-drive or synthetic conditions where lidar data is involved. Publicly available datasets can be enhanced by means of injecting objects into the scenes. Real-world objects point cloud can be injected into synthetic lidar data or synthetic point cloud object can be injected into the real world to create multiple scenarios test data to check the efficacy of \acrshort{had} systems. This approach is different from the existing methods in various sense. Use of 3D geometric information of points is used to extract an object without the need of 3D Point Cloud Segmentation models. Original rays directions are used to recreate the point cloud of the injected object, which makes the approach reliable in the recreation of new point cloud scenarios.
